% !TeX spellcheck = en_US
% !TEX TS-program = pdflatex
% !TEX encoding = IsoLatin

%% Version 4x3 und 16x9  2.2 02.01.2014

%Based on ETH official latex template
%Modified for ASL by Alvaro Estandia (09.02.2015) (ealvaro@student.ethz.ch)

% ==== wrapper class ==========================================================
\documentclass[% wrapper-class ETHpres option inspite of aspectratio for beamer-classe
    fourtothree=true, % true (default) -- 4:3-format, false -- 16:9-format
    DepLogo=true     % true -- use deplogo_13.pdf, false (default) 
                      %         do not use deplogo_13.pdf for footer
    ]{ETHpres}

% ==== misc: you may use or not ===============================================
%\usepackage{graphicx}   % for including figures
%%\graphicspath{{pictures/}}
%\usepackage{tabularx}   % for special table environment (tabularx-table)
%\usepackage{booktabs}   % for table layout
%\usepackage{natbib}     % for bibliography with astron-style
%\bibliographystyle{astron}
%\usepackage{siunitx}    % to use for international units in the real world
%\usepackage[
%    colorlinks=true, linkcolor=white, urlcolor=white, % this is special for this presentation here to get the toc in white
%    hypertexnames=false,% for correct links (duplicate-error solution)
%	setpagesize=false,  % necessary in order to not change text-/paperformat for the document
%	pdfborder={0 0 0},  % removes border around links
%	pdfpagemode=FullScreen,% open pdf in full screen mode
%    pdfstartview=Fit    % fit page to pdf viewer
%]{hyperref}% all links stay black and are thus invisible

%----- My Packages
\titlespacing{\section}{0pt}{-25pt}{0pt}
\titlespacing{\subsubsection}{0pt}{5pt}{0pt}

\usepackage{amsmath}
\usepackage{caption}
\usepackage{comment}
\usepackage[makeroom]{cancel}


\setitemize{noitemsep,topsep=0pt,parsep=0pt,partopsep=0pt}

%\usepackage{soul} Highligh text \hl{Some Text}
\usepackage{media9} %Add videos

%ASL Packages
\usepackage[numbers]{natbib}
\usepackage{enumitem}
\usepackage{units}

\usepackage{isomath}
\renewcommand{\vec}{\vectorsym}
\newcommand{\mat}{\matrixsym}

% Tikz drawings
\usepackage{tikz}
\usetikzlibrary{calc}

% ==== language ================================================================
\usepackage[latin1]{inputenc}
%\usepackage[utf8]{inputenc}
% English
\usepackage[english]{babel}
\AtBeginDocument{\renewcaptionname{english}{\contentsname}{ }}% toc-name
%% Deutsch
%\usepackage[ngerman]{babel}
%\AtBeginDocument{\renewcaptionname{ngerman}{\contentsname}{ }}% toc-name

% ==== choose the basic color for your presentation ===========================
% colorbar-color
\colorlet{firstcolor}{ETHc} % see pages 2  and 3 of this sample presentation
% bachground color titlepage
\colorlet{secondcolor}{ETHc} % see pages 2  and 3 of this sample presentation

% === fill in first information for the presentation ==========================
\newcommand*{\ETHtitle}{Closed-loop multi-sensor SLAM using factor graphs for fixed-wing UAV.}
\newcommand*{\ETHauthor}{Adam Radomski}
\newcommand*{\ETHdate}{16.06.2017}


\begin{document}
% =========== begin of titlepage ============
\ETHtitelbild\textcolor{white}{\large\textbf{\ETHtitle}}\\~\newline\hspace{6mm}\normalsize%
%%
% ==== start here with the text on the titlepage
\textcolor{white}{
\textbf{\ETHauthor}\\ \\
Master Thesis\\
Supervised by Timo Hinzmann, Thomas Schneider}\\
%%


\ETHslide
\section*{Motivation}
Develop localization framework which can simultaneously:
\begin{itemize}
	\item[\ETHitem] Estimate local navigation solution with minimal latency
	\item[\ETHitem] Find optimal solution given all the measurements
\end{itemize}

\clearpage

\ETHslide
\section*{Approach}
Splitting the problem into short and long term problems (FIX)

\begin{center}
\includegraphics[width=0.5\textwidth]{TikZ_drawings/Simple_STS_and_LTS_diagram/Simple_STS_and_LTS.pdf}\\
\end{center}


% STS marginalizing old states
% LTS storing in a map, inputting landmarks into STS and doing loop closure.

%\begin{tikzpicture}
% comment it out at the end
%\draw[help lines] (0,0) grid(20,20);

% STS rectangle
\coordinate (TLSTS) at (4,16);
\coordinate (BRSTS) at (12,12) {};

% LTS rectangle
\coordinate (TLLTS) at (4,10);
\coordinate (BRLTS) at (12,6);

% Data feeder rectangle
\coordinate (TLDF) at (0,15);
\coordinate (BRDF) at (2,13) {};

% Draw rectangles and their names in corners
\draw [ultra thick, rounded corners, blue] (TLSTS) rectangle (BRSTS);
\draw [ultra thick, rounded corners, blue] (TLLTS) rectangle (BRLTS);
\node [below right] at (TLSTS) {Short Term Smoother};
\node [below right] at (TLLTS) {Long Term Smoother};

% Draw Vision feeder
\draw [ultra thick, rounded corners, blue] (TLDF) rectangle (BRDF);
\node [below right] at (TLDF) {Data input};

% Draw data flow
\draw [thick, ->] ($(BRDF) + (0,1.5)$) -- ($(TLSTS) - (0,1.5)$) node [above left] {Vision};
\draw [thick, ->] ($(BRDF) + (0,1)$) -- ($(TLSTS) - (0,2)$) node [above left] {IMU};
\draw [thick, ->] ($(BRDF) + (0,0.5)$) -- ($(TLSTS) - (0,2.5)$) node [above left] {GPS};
\draw [thick, ->] ($(BRDF) + (0,0.5)$) -- ($(BRDF) + (1,0.5)$)
  -- ($(BRDF) + (1,-5)$) --  ($(TLLTS) - (0,2)$)  node [above left] {GPS};



%% TODO add it below \end
%\caption{Do not forget!
%Make it explicit enough that readers
%can figure out what you are doing.}
\end{tikzpicture}


    
    

\clearpage

\ETHslide
\section*{Work done so far}
%probably split it into two slides and show what is done
Backbone of the localization framework
\begin{itemize}
	\item[\ETHitem] Short Term Smoother 
		\begin{itemize}
			\item building a full factor graph given sensor data
			\item estimating position and passing data to LTS
		\end{itemize}
\end{itemize}		
		
		
		
\clearpage

\ETHslide
\section*{Work done so far}
Backbone of the localization framework
\begin{itemize}		
	\item[\ETHitem] Long Term Smoother
		\begin{itemize}
		 	\item building a map with the input data	
		 	\item "translating" the map to a factor graph
		 	\item optimizing the factor graph and updating data in the map
		\end{itemize}
\end{itemize}

\clearpage

\ETHslide		
\section*{Short and Long Term Smoother}
\begin{center}
\includegraphics[width=0.7\textwidth]{TikZ_drawings/STS_and_LTS/STS_and_LTS.pdf}\\
\end{center}

\clearpage

\ETHslide
\section*{Current challenges}
\begin{itemize}
	\item[\ETHitem] Reading landmarks from the map and translating them into a factor graph
	\item[\ETHitem] Inserting fixed landmarks into STS
\end{itemize}
HERE THE LANDMARK IDEA

\clearpage

\ETHslide
\section*{Future work}
\begin{itemize}
	\item[\ETHitem] 3-stage landmark initialization
	\begin{itemize}
		\item Stage 1: compute 3D landmark coordinate and initialize the feature as binary factor (state $x_k$ and $x_{k+1}$).
		% what is binary factor?
		\item Stage 2: formulate the feature re-projection factors connecting the 3D landmark state and pose.
		\item Stage 3: once uncertainty converges marginalize landmark state and switch back to binary factor formulation.
	\end{itemize}
 	\item[\ETHitem] Sliding-Window STS
	\begin{itemize}
		\item Reduce the STS problem to a sliding-window factor graph		
	\end{itemize} 	
\end{itemize}

\clearpage



% =========== begin of the standard page ============

\ETHslide
\section*{Overview}
\tableofcontents

\clearpage

\ETHslide
\section*{Adding a video}
\addcontentsline{toc}{section}{Adding a video}
%Example included because it took me very long to figure it out.
\includemedia[
		  width=0.8\textwidth,
		  height=0.45\textwidth,
		  activate=pageopen,
		  addresource=LittleDog.mp4,
		  flashvars={source=LittleDog.mp4}
		]{}{VPlayer.swf}\\
		\footnotesize{LittleDog walking over rough terrain (S. Schaal, ``The latest version of the LittleDog Robot,'' 2010. https://www.youtube.com/watch?v=nUQsRPJ1dYw)}

			
\clearpage


\ETHslide
\section*{Adding a video - Example Slide}
\addcontentsline{toc}{subsection}{Example Slide}
\begin{minipage}{0.6\textwidth}
	\begin{itemize}
		\item[\ETHitem] Point 1
		\item[\ETHitem] Point 2
		\begin{itemize}
			\item Point 1.1
			\item Point 1.2
		\end{itemize}
\end{itemize}
\end{minipage}
\begin{minipage}{0.39\textwidth}
	\centering
	\includegraphics[width=0.75\textwidth]{SlopeStdDev.png}\\
	\footnotesize{Caption}
\end{minipage}

\begin{comment}
	Add a video file
	
	\includemedia[
	 width=\textwidth,
	 height=0.55\textwidth,
	 activate=pageopen,
	 addresource=Video/BigDog.mp4,
	 flashvars={source=Video/BigDog.mp4}
			]{}{VPlayer.swf}
	\footnotesize{Caption}
	
	\end{comment}

\clearpage

\begin{comment}
References Slide
\ETHslide
\addcontentsline{toc}{section}{References}
\bibliographystyle{bibliography/IEEEtranN}
\tiny{
\bibliography{FILE}}
\clearpage
\end{comment}
\end{document}


